%!TEX TS-program = xelatex
\documentclass[]{friggeri-cv}
\usepackage{afterpage}
\usepackage{hyperref}
\usepackage{color}
\usepackage{xcolor}
\usepackage{smartdiagram}
\usepackage{fontspec}
% if you want to add fontawesome package
% you need to compile the tex file with LuaLaTeX
% References:
%   http://texdoc.net/texmf-dist/doc/latex/fontawesome/fontawesome.pdf
%   https://www.ctan.org/tex-archive/fonts/fontawesome?lang=en
%\usepackage{fontawesome}
\usepackage{metalogo}
\usepackage{dtklogos}
\usepackage[utf8]{inputenc}
\usepackage{tikz}
\usetikzlibrary{mindmap,shadows}
\hypersetup{
    pdftitle={},
    pdfauthor={},
    pdfsubject={},
    pdfkeywords={},
    colorlinks=false,           % no lik border color
    allbordercolors=white       % white border color for all
}
\smartdiagramset{
    bubble center node font = \footnotesize,
    bubble node font = \footnotesize,
    % specifies the minimum size of the bubble center node
    bubble center node size = 0.5cm,
    %  specifies the minimum size of the bubbles
    bubble node size = 0.5cm,
    % specifies which is the distance among the bubble center node and the other bubbles
    distance center/other bubbles = 0.3cm,
    % sets the distance from the text to the border of the bubble center node
    distance text center bubble = 0.5cm,
    % set center bubble color
    bubble center node color = pblue,
    % define the list of colors usable in the diagram
    set color list = {lightgray, materialcyan, orange, green, materialorange, materialteal, materialamber, materialindigo, materialgreen, materiallime},
    % sets the opacity at which the bubbles are shown
    bubble fill opacity = 0.6,
    % sets the opacity at which the bubble text is shown
    bubble text opacity = 0.5,
}

\addbibresource{bibliography.bib}
\RequirePackage{xcolor}
\definecolor{pblue}{HTML}{0395DE}

\begin{document}
\header{Göran}{Bäcklund}
      {Systemutvecklare / Dataingenjör}
      
% Fake text to add separator      
\fcolorbox{white}{gray}{\parbox{\dimexpr\textwidth-2\fboxsep-2\fboxrule}{%
.....
}}

% In the aside, each new line forces a line break
\begin{aside}
  \includegraphics[scale=0.10]{img/jag.png}
  \section{Address}
    Pärlagerkvistsväg 8 
    352 43 Växjö
    ~
  \section{Tel}
   076-345 03 62
    ~
  \section{Mail}
    \href{mailto:gbanm06@gmail.com}{\textbf{gbanm06}}
    ~
  \section{Git}
    \href{https://github.com/backlundtransform}{backlundtransform}
    ~
  % use  \hspace{} or \vspace{} to change bubble size, if needed
  \section{Programmering}
    \smartdiagram[bubble diagram]{
        \textbf{\hspace{3ex}C\#\hspace{3ex}},
        \textbf{PHP},
        \textbf{Java},
        
       \textbf{Typescript},
       \textbf{VB},
        \textbf{\vspace{-3ex}Python\vspace{-1ex}}
        

    }
    ~
  \section{Personliga egenskaper}
    \smartdiagram[bubble diagram]{
          \textbf{Analytisk},
        \textbf{Pedagogisk},
        \textbf{Nyfikenhet},
        \textbf{Ambitiös},
        \textbf{Strukturerad},
         \textbf{Kundfokus}
    }
\end{aside}
~
\section{Erfarenhet}
\begin{entrylist}
  \entry
    {jun 2015 - nu}
    {IT-konsult}
    {Zenta AB}
    {
Arbetar som konsult på en digitaliseringsbyrå: \newline 

 \textbf{DHI}, expertkonsult för utvecklingen av Future City Flow och dess relaterade projekt. Plattformarna analyserar regn och dess effekter på avloppssystem genom att integrera GIS-tjänster för att visualisera hydrologiska data i kartapplikationer. Jag var ansvarig för utvecklingen av ett dataanalysverktyg som utför beräkningar på tidsseriedata för att ta fram KPI-värden för tänkta åtgärder på ledningssystemen.
Jag har även utvecklat ett hydroövervakningssystem som används av en stor aktör inom kärnkraftsindustrin\newline

 \textbf{RISE},jag deltog i ett forskningsprojekt inom maskininlärning där vi samlade in data från sensorer installerade i glasbrukens smältugnar. Syftet var att använda denna data för att göra prediktioner om glaskvaliteten. För att bygga och träna modellerna använde vi oss av Scikit-learn och Keras\newline

  \textbf{Bryne}, utveckling av ett analysverktyg för att beräkna fluidity index för gjutformar\newline
  
  \textbf{Optigrow}, utveckling av en mobilapplikation som används för att övervaka processer på maskiner\newline

  \textbf{Atea logistic}, utveckling av en portal som integrerar med kundens interna säljstödssystem. En modul har utvecklats för att kunna söka i stora datamängder, som fungerar som prisjakt. Ravendb och Azure search används för att optimera sökningarna
 }
 \entry
    {jan 2021 - nu}
    {Projekt koordinator}
    {Mattecentrum}
    {Projekt koordinator för de fysiska räknestugorna i Växjö och styrelse ordförande för den lokala föreningen}
  \entry
    {mars 2015 - jun 2015}
    {Systemutvecklare}
    {Office IT-Partner}
    {Jag utvecklade en hybrid-mobilapplikation för företagets ärendehanteringssystem och hjälpte till att migrera databasen till deras nya system}
    \entry
    {dec 2013 - dec 2014}
    {IT pedagog}
    {Folkuniversitetet}
    {Deltidsanställd som IT-lärare inom Office-paketet på arbetsmarknadsutbildningen ”Korta vägen” för nyanlända akademiker. Arbetade även som cirkelledare på data-utbildning för seniorer och i kalkylering med Excel }
    \entry
    {jan 2010 - jan 2011}
    {Avhandlingsarbete inom teknisk fysik}
    {Fläktwoods AB}
    {Datasimuleringar (FEM) utfördes med Comsol multiphysics på en tvåstegs axialfläkt med sikte på att förbättra effektiviteten och minska ljudet från fläkten}
\end{entrylist}
\pagebreak 
\section{Utbildning}
\begin{entrylist}
   \entry
    {2014 - 2015}
    {Yrkesutbildning inom .Net utveckling}
    {Lexicon}
    {Kurser i webbutveckling .Net MVC, Entity framework SQL, Angularjs och bootstrap}
  \entry
    {2006 - 2011}
    {MSc i fysik}
    {Linnesuniversitet}
    {Fysik 135 hp, Matematik 82.5 hp, Datavetenskap 30 hp}
  \entry
    {2004 - 2005}
    {Teknisktbasår}
    {Forum Ystad}
    {Kurser i matematik, fysik, kemi och programmering}
\end{entrylist}


\begin{aside}
~
~
~
  \section{Industriell kunskap}
    \smartdiagram[bubble diagram]{
     \textbf{\hspace{2ex}Webb\hspace{2ex}},
     \textbf{IoT},
     \textbf{Industri 4.0},
     \textbf{\hspace{1ex}AI / ML \hspace{1ex}},
      \textbf{CAD/CFD},
    \textbf{\hspace{2ex}Mobil\hspace{2ex}},
 \textbf{\hspace{2ex}\vspace{-2ex}GIS\hspace{2ex}}
       
    }
    ~
  \section{Verktyg}
    \textbf{Azure}\includegraphics[scale=0.40]{img/5stars.png}
    \textbf{Vs}\includegraphics[scale=0.40]{img/5stars.png}
      \textbf{MsSQL}\includegraphics[scale=0.40]{img/4stars.png}
    \textbf{Git}\includegraphics[scale=0.40]{img/4stars.png}
    ~
  \section{Språk}
    \textbf{Svenska}\includegraphics[scale=0.40]{img/5stars.png}
    \textbf{Engelska}\includegraphics[scale=0.40]{img/4stars.png}
    ~
\end{aside}

\section{Volontärarbete}
\begin{entrylist}
   \entry
    {sep 2013 - nu}
    {Matematik lärare}
    {Mattecentrum}
    {Volontärarbete med att hjälpa elever med matematik uppgifter}
\end{entrylist}
\section{Tävlingar}
\begin{entrylist}
  \entry
    {sep 2013}
    {Hackathon}
    {East Sweden Hack}
    {Android app som kan hitta närmaste busshållplatser baserat på geografisk plats}
\end{entrylist}

\section{Certifieringar}
\begin{entrylist}
  \entry
    {dec 2015}
    {Ucommerce/Umbraco Certifiering}
    {Ucommerce}
        {}
\end{entrylist}
\section{Hobby}
{
Jag har sedan länge ett intresse för amatör astronomi och utvecklat applikationer som används för att visualisera stjärnhimlen}
\\
\end{document}
